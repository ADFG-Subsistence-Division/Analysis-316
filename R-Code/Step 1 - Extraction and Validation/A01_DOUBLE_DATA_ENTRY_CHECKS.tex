% Options for packages loaded elsewhere
\PassOptionsToPackage{unicode}{hyperref}
\PassOptionsToPackage{hyphens}{url}
%
\documentclass[
]{article}
\usepackage{amsmath,amssymb}
\usepackage{iftex}
\ifPDFTeX
  \usepackage[T1]{fontenc}
  \usepackage[utf8]{inputenc}
  \usepackage{textcomp} % provide euro and other symbols
\else % if luatex or xetex
  \usepackage{unicode-math} % this also loads fontspec
  \defaultfontfeatures{Scale=MatchLowercase}
  \defaultfontfeatures[\rmfamily]{Ligatures=TeX,Scale=1}
\fi
\usepackage{lmodern}
\ifPDFTeX\else
  % xetex/luatex font selection
\fi
% Use upquote if available, for straight quotes in verbatim environments
\IfFileExists{upquote.sty}{\usepackage{upquote}}{}
\IfFileExists{microtype.sty}{% use microtype if available
  \usepackage[]{microtype}
  \UseMicrotypeSet[protrusion]{basicmath} % disable protrusion for tt fonts
}{}
\makeatletter
\@ifundefined{KOMAClassName}{% if non-KOMA class
  \IfFileExists{parskip.sty}{%
    \usepackage{parskip}
  }{% else
    \setlength{\parindent}{0pt}
    \setlength{\parskip}{6pt plus 2pt minus 1pt}}
}{% if KOMA class
  \KOMAoptions{parskip=half}}
\makeatother
\usepackage{xcolor}
\usepackage[margin=1in]{geometry}
\usepackage{color}
\usepackage{fancyvrb}
\newcommand{\VerbBar}{|}
\newcommand{\VERB}{\Verb[commandchars=\\\{\}]}
\DefineVerbatimEnvironment{Highlighting}{Verbatim}{commandchars=\\\{\}}
% Add ',fontsize=\small' for more characters per line
\usepackage{framed}
\definecolor{shadecolor}{RGB}{248,248,248}
\newenvironment{Shaded}{\begin{snugshade}}{\end{snugshade}}
\newcommand{\AlertTok}[1]{\textcolor[rgb]{0.94,0.16,0.16}{#1}}
\newcommand{\AnnotationTok}[1]{\textcolor[rgb]{0.56,0.35,0.01}{\textbf{\textit{#1}}}}
\newcommand{\AttributeTok}[1]{\textcolor[rgb]{0.13,0.29,0.53}{#1}}
\newcommand{\BaseNTok}[1]{\textcolor[rgb]{0.00,0.00,0.81}{#1}}
\newcommand{\BuiltInTok}[1]{#1}
\newcommand{\CharTok}[1]{\textcolor[rgb]{0.31,0.60,0.02}{#1}}
\newcommand{\CommentTok}[1]{\textcolor[rgb]{0.56,0.35,0.01}{\textit{#1}}}
\newcommand{\CommentVarTok}[1]{\textcolor[rgb]{0.56,0.35,0.01}{\textbf{\textit{#1}}}}
\newcommand{\ConstantTok}[1]{\textcolor[rgb]{0.56,0.35,0.01}{#1}}
\newcommand{\ControlFlowTok}[1]{\textcolor[rgb]{0.13,0.29,0.53}{\textbf{#1}}}
\newcommand{\DataTypeTok}[1]{\textcolor[rgb]{0.13,0.29,0.53}{#1}}
\newcommand{\DecValTok}[1]{\textcolor[rgb]{0.00,0.00,0.81}{#1}}
\newcommand{\DocumentationTok}[1]{\textcolor[rgb]{0.56,0.35,0.01}{\textbf{\textit{#1}}}}
\newcommand{\ErrorTok}[1]{\textcolor[rgb]{0.64,0.00,0.00}{\textbf{#1}}}
\newcommand{\ExtensionTok}[1]{#1}
\newcommand{\FloatTok}[1]{\textcolor[rgb]{0.00,0.00,0.81}{#1}}
\newcommand{\FunctionTok}[1]{\textcolor[rgb]{0.13,0.29,0.53}{\textbf{#1}}}
\newcommand{\ImportTok}[1]{#1}
\newcommand{\InformationTok}[1]{\textcolor[rgb]{0.56,0.35,0.01}{\textbf{\textit{#1}}}}
\newcommand{\KeywordTok}[1]{\textcolor[rgb]{0.13,0.29,0.53}{\textbf{#1}}}
\newcommand{\NormalTok}[1]{#1}
\newcommand{\OperatorTok}[1]{\textcolor[rgb]{0.81,0.36,0.00}{\textbf{#1}}}
\newcommand{\OtherTok}[1]{\textcolor[rgb]{0.56,0.35,0.01}{#1}}
\newcommand{\PreprocessorTok}[1]{\textcolor[rgb]{0.56,0.35,0.01}{\textit{#1}}}
\newcommand{\RegionMarkerTok}[1]{#1}
\newcommand{\SpecialCharTok}[1]{\textcolor[rgb]{0.81,0.36,0.00}{\textbf{#1}}}
\newcommand{\SpecialStringTok}[1]{\textcolor[rgb]{0.31,0.60,0.02}{#1}}
\newcommand{\StringTok}[1]{\textcolor[rgb]{0.31,0.60,0.02}{#1}}
\newcommand{\VariableTok}[1]{\textcolor[rgb]{0.00,0.00,0.00}{#1}}
\newcommand{\VerbatimStringTok}[1]{\textcolor[rgb]{0.31,0.60,0.02}{#1}}
\newcommand{\WarningTok}[1]{\textcolor[rgb]{0.56,0.35,0.01}{\textbf{\textit{#1}}}}
\usepackage{graphicx}
\makeatletter
\def\maxwidth{\ifdim\Gin@nat@width>\linewidth\linewidth\else\Gin@nat@width\fi}
\def\maxheight{\ifdim\Gin@nat@height>\textheight\textheight\else\Gin@nat@height\fi}
\makeatother
% Scale images if necessary, so that they will not overflow the page
% margins by default, and it is still possible to overwrite the defaults
% using explicit options in \includegraphics[width, height, ...]{}
\setkeys{Gin}{width=\maxwidth,height=\maxheight,keepaspectratio}
% Set default figure placement to htbp
\makeatletter
\def\fps@figure{htbp}
\makeatother
\setlength{\emergencystretch}{3em} % prevent overfull lines
\providecommand{\tightlist}{%
  \setlength{\itemsep}{0pt}\setlength{\parskip}{0pt}}
\setcounter{secnumdepth}{-\maxdimen} % remove section numbering
\ifLuaTeX
  \usepackage{selnolig}  % disable illegal ligatures
\fi
\usepackage{bookmark}
\IfFileExists{xurl.sty}{\usepackage{xurl}}{} % add URL line breaks if available
\urlstyle{same}
\hypersetup{
  pdftitle={A01\_DOUBLE\_DATA\_ENTRY\_CHECKS - (316) NPS Ambler Comprehensive},
  pdfauthor={Jesse Coleman},
  hidelinks,
  pdfcreator={LaTeX via pandoc}}

\title{A01\_DOUBLE\_DATA\_ENTRY\_CHECKS - (316) NPS Ambler
Comprehensive}
\author{Jesse Coleman}
\date{2025-08-15}

\begin{document}
\maketitle

A01\_DOUBLE\_DATA\_ENTRY\_CHECKS - (316) NPS Ambler Comprehensive

\section{Double data entry checking}\label{double-data-entry-checking}

Execute double data entry checking routines on defined record types.

\subsection{Changelog}\label{changelog}

\begin{itemize}
\tightlist
\item
  Programmer: D.S.Koster\\
\item
  Date: 05/21/2024\\
\item
  Change Description: Updated organization and formatting to be more
  consistent and provide better feedback/output.\\
\item
  Template Update {[}Y\textbar N{]}: Y\\
\item
  One-Off: No-include in all future comprehensive templates
\end{itemize}

\subsection{Input data}\label{input-data}

\begin{itemize}
\tightlist
\item
  dfgjnusql-db71p/Sub\_SDS.dbo.xref\_community\_master\\
\item
  dfgjnusql-db71p/Sub\_SDS.dbo.vw\_resAndMarketRes\\
\item
  dfgjnusql-db71p/Sub\_SDS.dbo.META\_RECTYPE\_SPEC\\
\item
  dfgjnusql-db71p/Sub\_SDS.dbo.META\_DEF\_DISPLAY\_GROUPS\\
\item
  dfgjnusql-db71p/Sub\_SDS.dbo.META\_DATA\_DEFINITION
\item
  dfgjnusql-db71p/Sub\_SDS.dbo.vw\_definedRecordTypes\\
\item
  dfgjnusql-db71p/Sub\_SDS.dbo.vw\_sample\\
\item
  dfgjnusql-db71p/Sub\_SDS.dbo.SP\_GET\_DATA\_BY\_RECTYPE\_FULL\\
\item
  dfgjnusql-db71p/Sub\_SDS.dbo.vw\_definedColumns
\end{itemize}

\emph{NOTE:} No files from file-system are read in.

\subsection{Output data}\label{output-data}

Output files are written dynamically according to the defiend record
types in the SDS. You will get one file for each defined record type
having errors. The form of that will be: REC00\_DDErrs

\begin{itemize}
\tightlist
\item
  02 - Error Checking/\$\{dataSetName\}.csv
\end{itemize}

\subsection{Background}\label{background}

A core part of the quality assurance process with subsistence household
surveys is the entry of each survey twice by two different people. This
is a common practice employed when complex data entry is required, it
ensures the most accurate possible entry and helps to avoid issues with
handwriting interpretation as well.

After data entry version 1 and 2 have been completed, this file can be
run to identify records where data entry errors are present.

\subsection{Checklist}\label{checklist}

\begin{itemize}
\tightlist
\item
  Update `Author' to your name\\
\item
  Update the project information in `title' to the current project\\
\item
  Update the development log with any changes you've made to the
  template file, including your name\\
\item
  Evaluate each file produced (by record type) and verify that both
  versions match and are correct\\
\item
  All corrections have been updated directly in the SDS\\
\item
  All errors have been addressed\\
\item
  A final run of this file has occurred with no errors present.
\end{itemize}

\subsection{Additional Information}\label{additional-information}

This file uses some tricks. The first is that, we create a `vector' of
elements to iterate over. This is done both with recordTypeCD and
communty. However, the main trick is the shortened version. Note that in
{[}1{]} below, we use the \$ operator to reference just the desired
column before concatenating it c() into a list.

Second, we dynamically create a summarize function by first creating a
list of column names line {[}1{]} below then, in line {[}2{]} we use the
!!! operator in front of the list in our `group by' This way, we don't
have to know ahead of time what those columns are and instead bring them
straight out of the database. Note that the syms() function is essential
here otherwise, the !!! will not work as advertised.

\begin{Shaded}
\begin{Highlighting}[]
\NormalTok{ [}\DecValTok{1}\NormalTok{]   colNameList }\OtherTok{\textless{}{-}} \FunctionTok{syms}\NormalTok{(}\FunctionTok{c}\NormalTok{(}\FunctionTok{dbGetQuery}\NormalTok{(conn, checkColSQL)}\SpecialCharTok{$}\NormalTok{colName))}
\NormalTok{       tAggDataSet }\OtherTok{\textless{}{-}}\NormalTok{ tDataSet }\SpecialCharTok{\%\textgreater{}\%}  
\NormalTok{ [}\DecValTok{2}\NormalTok{]      }\FunctionTok{group\_by}\NormalTok{(., projID, studyear, communty, strata, HHID, }\SpecialCharTok{!!!}\NormalTok{colNameList) }\SpecialCharTok{\%\textgreater{}\%}
          \FunctionTok{summarize}\NormalTok{(., }\AttributeTok{version=}\FunctionTok{sum}\NormalTok{(version, }\AttributeTok{na.rm=}\ConstantTok{TRUE}\NormalTok{))  }
\end{Highlighting}
\end{Shaded}

\subsection{Required libraries}\label{required-libraries}

\begin{itemize}
\tightlist
\item
  tidyVerse\\
\item
  odbc\\
\item
  rio\\
\item
  knitr
\end{itemize}

\section{Data processing}\label{data-processing}

\subsection{Initialize environment}\label{initialize-environment}

Working on project 316 NPS Ambler Comprehensive - 2024

\subsection{Extract Codesets \& Sampling
information}\label{extract-codesets-sampling-information}

Extracts sampling information and codesets for use with this analysis.
Codesets may be fundamentally identical to those used in other projects,
however these are replicated per-project to ensure full-documentation of
the system state when analysis was run.

\emph{NOTE:} Codesets are extracted directly from the database because
this step of analysis is often run BEFORE the GET\_SDS\_DATA has been
run.

\subsubsection{Extract lookup codes}\label{extract-lookup-codes}

SQL Extract:SELECT communty, commname FROM dbo.xref\_community\_master,
906 records loaded.

SQL Extract:SELECT resource, resName FROM dbo.vw\_resAndMarketRes, 2856
records loaded.

SQL Extract:SELECT recordTypeCD, recordType FROM
dbo.META\_RECTYPE\_SPEC, 34 records loaded.

SQL Extract:SELECT D.projID, D.studyear, D.communty, G.displayGroup,
G.displayGroupDesc, G.pageName FROM dbo.META\_DEF\_DISPLAY\_GROUPS G
LEFT JOIN dbo.META\_DATA\_DEFINITION D ON G.dataDefinitionID =
D.dataDefinitionID WHERE D.projID = 316 AND D.studyear = 2024, 36
records loaded.

SQL Extract:SELECT DISTINCT {[}communty{]}, {[}recordTypeCD{]} FROM
{[}vw\_definedRecordTypes{]} WHERE {[}projID{]} = 316 AND {[}studyear{]}
= 2024 AND definedRecordTypeCD is not null, 17 records loaded.

SQL Extract:SELECT proj\_id AS projID, studyear, communty, strata,
strataName FROM dbo.vw\_sample\\
WHERE (proj\_id = 316 AND studyear = 2024), 1 records loaded.

\subsection{Execute double data entry
checks}\label{execute-double-data-entry-checks}

DDE Check - Record Type: 00 - Household information returned no results
(V1==V2).

DDE Check - Record Type: 01 - Person information returned no results
(V1==V2).

DDE Check - Record Type: 03 - Retained Commercial Harvests returned no
results (V1==V2).

DDE Check - Record Type: 04 - Salmon harvests (non commercial) returned
no results (V1==V2).

DDE Check - Record Type: 06 - Non-salmon Fish harvests (non commercial)
returned no results (V1==V2).

DDE Check - Record Type: 08 - Marine invertebrate harvests (non
commercial) returned no results (V1==V2).

DDE Check - Record Type: 10 - Large Land Mammals returned no results
(V1==V2).

DDE Check - Record Type: 12 - Marine Mammals returned no results
(V1==V2).

DDE Check - Record Type: 14 - Small Land mammals returned no results
(V1==V2).

DDE Check - Record Type: 15 - Birds and Eggs returned no results
(V1==V2).

DDE Check - Record Type: 17 - Vegetation returned no results (V1==V2).

DDE Check - Record Type: 201 - Food Security returned no results
(V1==V2).

DDE Check - Record Type: 23 - Household Employment returned no results
(V1==V2).

DDE Check - Record Type: 24 - Other sources of income returned no
results (V1==V2).

DDE Check - Record Type: 300 - Survey summary comments returned no
results (V1==V2).

DDE Check - Record Type: 376 - Resource health and safety returned no
results (V1==V2).

DDE Check - Record Type: 66 - Assessments returned no results (V1==V2).

\subsubsection{Write CSV}\label{write-csv}

Write out each dataset containing errors.

End of Double Data Entry Checking script.

\end{document}
